\documentclass{article}
\usepackage[utf8]{inputenc}
\usepackage[a4paper]{geometry}
\geometry{
    top=20mm,
    left=25mm,
    right=25mm,
    bottom=25mm
}

\usepackage{fontspec}
\setmonofont[Contextuals={Alternate}]{Jetbrains Mono}[Scale=MatchLowercase]

\usepackage{minted}
\setminted{
    linenos,
    frame=single
}

\usepackage{hyperref}

\title{Rust \& Typestates}
\author{José Duarte}
\date{\today}

\begin{document}
\maketitle

% \begin{abstract}
%     Typestates are a mechanism that enables developers to write stricter and less error-prone APIs,
%     their main characteristic is the leverage of the type system to aid in managing state.
%     Before its 1.0 release, Rust had typestates in the language.
%     The feature was removed due to ""
% \end{abstract}

\section*{Introduction}

\subsection*{What are typestates?}

In a nutshell, typestates can be thought of a way to constrain APIs as the program state evolves.
More formally, typestates belong to the behavioral types category and are built on the idea of lifting state to the type level,
since state becomes part of the type system, the compiler will be able to reason about state,
effectively helping the developer track state and validate certain assumptions.

\subsection*{Why are typestates useful?}

Diving deeper on why do typestates help the developer, I provide a simple yet classic example.
Consider a stream, whether it be a file or a socket, to be read, the stream must first be open,
only then can it be read and finally it must be closed.

\begin{minted}{java}
Scanner s = new Scanner(System.in); // open the stream
s.nextLine();                       // read
s.close();                          // close the stream
s.nextLine();                       // IllegalStateException
\end{minted}

In the example above the developer tries to read a line after closing the stream,
this yields a \texttt{IllegalStateException} since you cannot read from a closed source.

The fact that this code compiles without warnings (even when \texttt{-Xlint:all} is used) is problematic,
since the error can only be caught at runtime.
While the presented example is simple,
code running in production is not,
and code paths that raise runtime errors may be untested until it reaches the hands of the user.

Using typestates solves the above problem by establishing a distinction between the open and closed \texttt{Scanner},
consider the following example:

\begin{minted}{java}
Scanner[Open] s = new Scanner(System.in);   // open the stream
s.nextLine();                               // read
Scanner[Closed] s = s.close();              // close the stream
s.nextLine();                               // compile-time error
\end{minted}

The compiler is now able to provide the developer with an error at compile-time since it now knows that the \texttt{Scanner} is closed and thus,
it does not have a function \texttt{nextLine}.

\subsection*{Why Rust?}

One of Rust's core values is safety,
such value manifests itself in the form of memory safety,
and the provision of tools to prevent concurrency problems.
The mindset is also imbued in the community,
an example would be Sealed Rust\footnote{\url{https://ferrous-systems.com/blog/sealed-rust-the-plan/}},
an effort to bring Rust to the safety critical domain.

\subsubsection*{Alias Control}

Besides the pro-safety mentality of Rust,
another key detail which makes Rust a solid candidate for typestates is its alias control.

By definition, typestates are incompatible with aliasing,
this is due to the fact that if an object is being used by $N$ clients,
when a client mutates an object, all other clients guarantees are broken.
Back to the stream example, if a client closes the stream,
all other clients may crash since they may try to read from a stream which is now closed.

While Rust cannot enforce a truly linear typesystem (in which objects must be used \emph{exactly once}),
it is able to enforce an affine type system (or \emph{at most once} usage), as demonstrated bellow.

\begin{minted}{rust}
let x = 0
let y = x;
println!("{}", x); // error, value moved in line 2
\end{minted}

\section*{Typestating Rust}

Rust previously had a typestate system which was removed in version 0.4 since it "did not pull its own weight".
The idea I propose does is not aimed at bringing back the old typestate system but rather at leveraging Rust's type system to allow for typestated objects.
In short, I propose a DSL built with procedural macros which takes advantage of Rust's generics and affine type system capabilities.

\subsection*{The Typestate Pattern}

As of now, it is possible for a developer to make use of typestates in Rust,
the key idea is to write functions which consume the current object and return a new object,
the function signature should be similar to \mintinline{rust}{fn transition(self: OldState, args...) -> NewState}.

Each state can then carry information regarding the current computation, or not, and simply be a 0-sized structure.
States can be grouped into sets using traits,
these sets can be further restricted using the sealed trait pattern\footnote{\url{https://rust-lang.github.io/api-guidelines/future-proofing.html\#sealed-traits-protect-against-downstream-implementations-c-sealed}}.
An example implementation of the typestate pattern is available in \url{https://github.com/rustype/http-parser}.

% \begin{minted}{rust}
% trait StateTrait {}
% struct T<State: StateTrait> {
%     state: State
% }
% struct S1;
% impl StateTrait for S1 {}
% impl T<S1> {
%     fn transition(self) -> T<S2> {...}
% }
% struct S2;
% impl StateTrait for S2 {}
% impl T<S2> {
%     fn transition(self) -> Result<T<S2>> {...}
% }
% \end{minted}

\subsection*{Proposal}

The code necessary to implement the typestate pattern requires a lot of boilerplate,
furthermore, we want to be able to prove typestate properties (e.g. that all states are productive).
The former challenge can be solved through the use of macros,
and the latter is not novel in the literature,
to address it we just need to extract the state machine from the code and apply existing algorithms.

However, to do so without writing a static checker from the ground up we are required to use Rust's procedural function-like macros, these enable the creation of a DSL for typestates, thus solving both the boilerplate problem and the requirement for the ability to prove certain properties.

Currently, a prototype is under development, the development of such prototype falls under the scope of my MSc thesis, which is centered around the topic of typestates and Rust.
The prototype can be found in \url{https://github.com/rustype/typestate-rs}.

\begin{minted}{rust}
    typestate! {
        struct Drone {x: f32, y: f32}
        fn ping_coords(&self) -> (f32, f32);
        state Idle [last_landing: Time] {
            transition take_off(self) -> Hovering;
        }
        state Hovering {
            fn take_picture(&self, dst: &str);
            transition land(self) -> Idle;
        }
    }
\end{minted}

\end{document}
