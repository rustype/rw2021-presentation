% !TEX root = ./main.tex
\documentclass{beamer}
\usetheme{metropolis}
\title{Typestates \& Rust}
\date{28th March 2021}
\author{José Duarte}
\institute{FCT NOVA}

\usepackage{listings}
\lstset{
    basicstyle=\small\ttfamily,
    frame=single,
    % numberstyle=\tiny,
    % numbers=left
}

\begin{document}
\maketitle
% \section{Introduction}
\begin{frame}
    \frametitle{What Are Typestates?}
    Typestates capture aspects of an object's state.
    % e.g. closed vs open file
    Approaches to typestates can be:
    \begin{itemize}
        \item \textbf{State-Based Designs} - checking state during runtime.
        \item \textbf{Typestate Checkers} - static checkers that verify code for correct object usage.
        \item \textbf{Language-Based} - design a language which supports typestates as first-class citizens. % e.g. Plaid
    \end{itemize}
\end{frame}

\begin{frame}[fragile]
    \frametitle{Why Are Typestates Useful?}
    Consider the \texttt{File} example:
    \begin{lstlisting}
Scanner s = new Scanner(System.in);
s.nextLine();   // ok
s.close();      // ok
s.nextLine();   // IllegalStateException
    \end{lstlisting}
\end{frame}

\begin{frame}[fragile]
    \frametitle{Why Are Typestates Useful?}
    What we really want:
    \begin{lstlisting}
| s.nextLine();
| ^^^^^^^^^^^^
= error: `s` was closed in line 3
    \end{lstlisting}
\end{frame}

\begin{frame}[fragile]
    \frametitle{Why Are Typestates Useful?}
    To achieve it, typestates lift state to the type level:
    \begin{lstlisting}
Scanner[Open] s = new Scanner(System.in);
s.nextLine();
Scanner[Closed] s = s.close();
s.nextLine();
    \end{lstlisting}
    The last line now fails to typecheck since \texttt{Scanner[Closed]}
    does not have function \texttt{nextLine}.
\end{frame}

\begin{frame}
    \frametitle{Typestates \& Rust}

    Typestates are not new to Rust.
    \begin{itemize}
        \item There are several blog posts on the subject.
        \item The \textbf{Embedded Rust Book} makes use of them to model peripherals.
        \item The \texttt{state\_machine\_future} crate leverages typestates to provide type safe \texttt{Future}s.
    \end{itemize}
\end{frame}

\begin{frame}
    \frametitle{Typestates \& Rust - Problems}

    \begin{itemize}
        \item Typestates require a lot of boilerplate.
        \item Existing crates either do not provide compile-time errors or are limited in scope.
    \end{itemize}
\end{frame}

\begin{frame}[fragile]
    \frametitle{Typestates \& Rust - Boilerplate}

    \begin{lstlisting}
    boilerplate demo
    \end{lstlisting}
\end{frame}

\begin{frame}[fragile]
    \frametitle{How Do We Solve It?}
    Macros! They are designed to write the boilerplate for us!
\end{frame}

\begin{frame}[fragile]
    \frametitle{The solution}
    Currently, we have implemented a small proof of concept which allows a user to define a typestated structure as:
    \begin{lstlisting}
typestate!(
    strict Drone [Idle, Hovering, Flying] {
        x: f32,
        y: f32
    }
);
    \end{lstlisting}
\end{frame}

\begin{frame}[fragile]
    \frametitle{The Solution}
    The previous code will generate:
    \begin{itemize}
        \item A \texttt{Drone} structure.
        \item The state structures \texttt{Idle, Hovering, Flying}.
        \item Some relevant \texttt{trait}s and \texttt{impl}s.
    \end{itemize}
\end{frame}

\begin{frame}[fragile]
    \frametitle{Limitations \& Moving Forward}
    \texttt{typestate!} is implemented with \texttt{macro\_rules},
    this has the downsides of being limited and barely readable.

    Ideally we want to write a full-fledged DSL,
    reducing the amount of code the developer has to write.
\end{frame}

\begin{frame}[fragile]
    \frametitle{A Peek Into The Future}
    \begin{lstlisting}
ts! {
    struct Drone { x: f32, y: f32 }

    fn ping_coordinates(&self) -> (f32, f32);

    state Idle [last_landing: Time] {
        transition take_off(self) -> Hovering;
    }

    state Hovering {
        fn take_picture(&self, dst: &str);
        transition land(self) -> Idle;
    }
}
    \end{lstlisting}
\end{frame}
\end{document}